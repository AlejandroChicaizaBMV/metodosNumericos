\documentclass[12pt]{article}
\usepackage[spanish]{babel}
\usepackage{geometry}
\geometry{a4paper, margin=1in}
\usepackage{graphicx}
\usepackage{xcolor}
\usepackage{titlesec}
\usepackage{parskip}
\usepackage{multicol}
\usepackage{cite}
\usepackage{listings}
\usepackage{color}
\usepackage{amsmath}
\usepackage{enumitem}

\lstset{
  language=Python,
  basicstyle=\ttfamily\small,
  keywordstyle=\color{blue},
  commentstyle=\color{gray},
  stringstyle=\color{red},
  breaklines=true,
  showstringspaces=false
}


\definecolor{highlight}{RGB}{255, 255, 0}

\titleformat{\section}{\normalfont\Large\bfseries}{\thesection}{1em}{}
\titleformat{\subsection}{\normalfont\large\bfseries}{\thesubsection}{1em}{}

\begin{document}

% Logos
\begin{minipage}{0.45\textwidth}
    \includegraphics[width=0.4\textwidth]{inFiles/Figures/epnLogo.jpg}
\end{minipage}
\hfill
\begin{minipage}{0.45\textwidth}
    \raggedleft
    \includegraphics[width=0.4\textwidth]{inFiles/Figures/FIS_logo.jpg}
\end{minipage}

\vspace{0.5cm}

% Títulos principales
\begin{center}
    \textbf{ESCUELA POLITÉCNICA NACIONAL}\\[0.2cm]
    \textbf{FACULTAD DE INGENIERÍA DE SISTEMAS}\\[0.2cm]
    \textbf{INGENIERÍA EN CIENCIAS DE LA COMPUTACIÓN}
\end{center}

\vspace{0.5cm}
\hrule
\vspace{0.5cm}

% Datos principales
\noindent\textbf{PERÍODO ACADÉMICO:} 2025-A\\[0.2cm]
\noindent\textbf{ASIGNATURA:} ICCD412 Métodos Numéricos \hfill \textbf{GRUPO:} GR2\\[0.2cm]
\noindent\textbf{TIPO DE INSTRUMENTO:} Tarea 9\\[0.2cm]
\noindent\textbf{FECHA DE ENTREGA LÍMITE:} {25/05/2025}\\[0.2cm]
\noindent\textbf{ALUMNO:} {Sebastián Chicaiza}

\vspace{0.5cm}
\hrule
\vspace{1cm}


% Secciones
\section*{TEMA}

\begin{center}
    \Large\textbf{Polinomio de Taylor}
\end{center}
\vspace{0.5cm}

\section*{OBJETIVOS}
\begin{itemize}
    \item {Poder comprender el concepto del polinomio de Taylor como una herramienta para aproximar funciones}
    \item {Lograr analizar el comportamiento del error por truncamiento asociado a la aproximación de Taylor}
\end{itemize}
\vspace{0.5cm}
\section*{MARCO TEÓRICO}
El teorema de Taylor es de gran valor en el estudio de los métodos numéricos, más específicamente
la serie de Taylor proporciona un medio para predecir el valor de una función
en un punto en términos del valor de la función y sus derivadas en otro punto.
El teorema establece que cualquier función suave puede aproximarse por un polinomio.

Serie de taylor:
\[\sum_{k = 0}^{n} \frac{f^{(k)}(x_0)}{k!} (x - x_0)^{k}\]

Se añade para terminar un término residual para considerar todos los términos desde el \(k + 1\) hasta el infinito

Término residual:

\[R_n(x) = \frac{f^{(n+1)}(\xi(x))}{(n+1)!} (x - x_0)^{n+1}\]

Donde el subindice \(n\) indica que éste es el residuo de la aproximación de n-ésimo orden
y \(\xi(x)\) es un valor de x que se encuentra en algún punto entre \(x_i\) y \(x_{i+1}\). \cite{chapra2011metodos}   

\section*{DESARROLLO}
Dada la función \(e^{-x}\) y \(x_0 = 1\). Determinar la función aproximada \( f(x) = P_n(x) + R_n(x)\) con \(n = 3\).

Primero sacamos las derivadas ya que las vamos a tener que usar, en este caso ya que \(n = 3\) sacamos hasta la derivada de orden 4 ya que \(R_n(x)\) tiene una derivada a la \(n+1\).

\[
    \begin{aligned}
         f(x) &= e^{-x} \\
         f^{'}(x) &= -e^{-x} \\
         f^{''}(x) &= e^{-x} \\
         f^{(3)}(x) &= -e^{-x} \\
         f^{(4)}(x) &= e^{-x} 
    \end{aligned}
\]

Sacamos el polinomio de Taylor:

\[
    \begin{aligned}
        P_n(x) &= \sum_{k = 0}^{n} \frac{f^{(k)}(x_0)}{k!} (x - x_0)^{k} \\
               &= f(x_0) + f^{'}(x_0) (x-x_0) + \frac{f^{''}(x_0)}{2!} (x - x_0)^{2} + \frac{f^{(3)}(x_0)}{3!} (x - x_0)^{3} \\  
               &= e^{-1} - e^{-1} (x-1) + \frac{e^{-1}}{2} (x-1)^{2} - \frac{e^{-1}}{6} (x - 1)^{3} \\
               &= e^{-1}\left[ 1 - (x-1) + \frac{1}{2} (x -1)^{2} - \frac{1}{6} (x -1)^{3}  \right] \\
               &= e^{-1}\left[ 2 - x + \frac{1}{2} (x -1)^{2} - \frac{1}{6} (x -1)^{3}  \right] \\
               &= e^{-1}\left[ 2 - x + \frac{1}{2} (x^{2} -2x +1) - \frac{1}{6} (x^{3} -3x^{2} + 3x - 1)  \right] \\
               &= e^{-1}\left[ 2 - x +  \frac{1}{2}x^{2} -x +\frac{1}{2} -  \frac{1}{6}x^{3} +\frac{1}{2}x^{2} -\frac{1}{2} x +\frac{1}{6}   \right] \\
               &= e^{-1}\left[  -  \frac{1}{6}x^{3} + x^{2} - \frac{5}{2}x   + \frac{8}{3}   \right] \\
    \end{aligned}
\]

Sacamos el término residual:
\[
    \begin{aligned}
        R_n (x) &= \frac{f^{(n+1)}(\xi(x))}{(n+1)!} (x - x_0)^{n+1} \\
                &= \frac{f^{(4)}(\xi(x))}{(4)!} (x - 1)^{4} \\
                &= \frac{e^{- \xi (x)}}{(4)!} (x - 1)^{4} \\
                &= \frac{e^{- \xi (x)}}{(4)!} (x-1)^{2} (x-1)^{2} \\
                &= \frac{e^{- \xi (x)}}{(4)!} (x^{2}-2x+1) (x^{2}-2x+1) \\
                &= \frac{e^{- \xi (x)}}{(4)!} (x^{4} -2x^{3} +x^{2} -2x^{3} + 4x^{2} -2x +  x^{2}-2x+1)\\
                &= \frac{e^{- \xi (x)}}{(4)!} (x^{4} -4x^{3} + 6x^{2} -4x +1) 
    \end{aligned}
\]

Finalmente sumamos los resultados para obtener \(f(x)\):
\[
    \begin{aligned}
    f(x) &= P_n(x) + R_n(x)\\
         &= e^{-1}\left[  -  \frac{1}{6}x^{3} + x^{2} - \frac{5}{2}x   + \frac{8}{3}   \right]+\frac{e^{- \xi (x)}}{(4)!} (x^{4} -4x^{3} + 6x^{2} -4x +1) \\
    \end{aligned}
\]

% \section*{CONCLUSIONES}
%\begin{itemize}
%    \item {La elección del método a utilizarse depende de la función que deseemos analizar.}
%    \item {El método de la secante al no requerir la derivada es una alternativa muy útil aunque no garantiza una converguencia.} 
%\end{itemize}

%\section*{RECOMENDACIONES}

%\begin{itemize}
%    \item Usar herramientas de graficación como geogebra.
%\end{itemize}

\renewcommand{\refname}{\MakeUppercase{REFERENCIAS}}
\bibliographystyle{IEEEtran}
\bibliography{inFiles/References/references.bib}


\end{document}
