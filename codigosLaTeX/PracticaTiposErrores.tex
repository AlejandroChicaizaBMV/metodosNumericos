\documentclass[12pt]{article}
\usepackage[spanish]{babel}
\usepackage{geometry}
\geometry{a4paper, margin=1in}
\usepackage{graphicx}
\usepackage{xcolor}
\usepackage{titlesec}
\usepackage{parskip}
\usepackage{multicol}
\usepackage{cite}
\usepackage{listings}
\usepackage{color}
\usepackage{amsmath}
\usepackage{enumitem}

\lstset{
  language=Python,
  basicstyle=\ttfamily\small,
  keywordstyle=\color{blue},
  commentstyle=\color{gray},
  stringstyle=\color{red},
  breaklines=true,
  showstringspaces=false
}


\definecolor{highlight}{RGB}{255, 255, 0}

\titleformat{\section}{\normalfont\Large\bfseries}{\thesection}{1em}{}
\titleformat{\subsection}{\normalfont\large\bfseries}{\thesubsection}{1em}{}

\begin{document}

% Logos
\begin{minipage}{0.45\textwidth}
    \includegraphics[width=0.4\textwidth]{inFiles/Figures/epnLogo.jpg}
\end{minipage}
\hfill
\begin{minipage}{0.45\textwidth}
    \raggedleft
    \includegraphics[width=0.4\textwidth]{inFiles/Figures/FIS_logo.jpg}
\end{minipage}

\vspace{0.5cm}

% Títulos principales
\begin{center}
    \textbf{ESCUELA POLITÉCNICA NACIONAL}\\[0.2cm]
    \textbf{FACULTAD DE INGENIERÍA DE SISTEMAS}\\[0.2cm]
    \textbf{INGENIERÍA EN CIENCIAS DE LA COMPUTACIÓN}
\end{center}

\vspace{0.5cm}
\hrule
\vspace{0.5cm}

% Datos principales
\noindent\textbf{PERÍODO ACADÉMICO:} 2025-A\\[0.2cm]
\noindent\textbf{ASIGNATURA:} ICCD412 Métodos Numéricos \hfill \textbf{GRUPO:} GR2\\[0.2cm]
\noindent\textbf{TIPO DE INSTRUMENTO:} Práctica1\\[0.2cm]
\noindent\textbf{FECHA DE ENTREGA LÍMITE:} {04/05/2025}\\[0.2cm]
\noindent\textbf{ALUMNO:} {Sebastián Chicaiza}

\vspace{0.5cm}
\hrule
\vspace{1cm}


% Secciones
\section*{TEMA}

\begin{center}
    \Large\textbf{Tipos de Errores}
\end{center}
\vspace{0.5cm}

\section*{OBJETIVOS}
\begin{itemize}
    \item {Identificar los tipos de errores junto con su significado.}
    \item {Lograr resolver problemas con los diferentes tipos de errores.}
\end{itemize}
\vspace{0.5cm}
\section*{MARCO TEÓRICO}

Los errores númericos surgen del uso de aproximaciones, por la necesidad de representar
operaciones y cantidades matématicas exactas.
Entre los tipos de errores tenemos a los errores por truncamiento que resultan de 
las aproximaciones como un procedimiento matemático exacto.
Los errores por redondeo se producen cuando se usan números que tienen un limite
de cifras significativas para representar números exáctos.
Para los dos tipos de errores, el valor del error está dado por:

\[error = Valor_{Exacto} - Valor_{Aproximado}\]\cite{chapra2011metodos}

Donde \(error\) se usa para representar el valor exacto del error. Esta forma de reprsentar el
valor del error es también llamada \textbf{error verdadero}.

Las diferentes formas de calcular el error:

\begin{itemize}
    \item {Error absoluto:}
    \[error_{absoluto} = \left|Valor_{Exacto} - Valor_{Aproximado} \right|\]
    \item {Error relativo:}
    \[error_{relativo} = \left|\frac{Valor_{Exacto} - Valor_{Aproximado}}{Valor_{Exacto}} \right|\]
    \item {Error Porcental:}
    \[error_{porcentual} = \left|\frac{Valor_{Exacto} - Valor_{Aproximado}}{Valor_{Exacto}} \right| \cdot 100\%\]
\end{itemize}

\vspace{0.5cm}

\section*{DESARROLLO}

\subsection*{Conjunto de Ejercicios}

\begin{enumerate}
    \item Calcule los errores absoluto y relativo en las aproximaciones de \(p\) por \(p^{*}\).
    \begin{enumerate}[label=\alph*.]
        \item \(p = \pi, p^{*} = \frac{22}{7}\)
        \[
        \begin{aligned}
            error_{absoluto} &= \left| p - p^{*}\right| \\
                             &= \left| \pi - \frac{22}{7} \right| \\
                             &= \left|-0.0012644892673496777\right| = 0.0012644892673496777 \\
                             &\approx 0.00126449
        \end{aligned}
        \]

        \[
        \begin{aligned}
            error_{relativo} &= \left| \frac{p - p^{*}}{p}\right| \\
                             &= \frac{0.00126449}{\pi } \\
                             &= 0.0004024996679805415 \\
                             &\approx 0.0004025
        \end{aligned}
        \]
        \item \(p = \pi, p^{*} = 3.1416\)
        \[
            \begin{aligned}
                error_{absoluto} &= \left| p - p^{*}\right| \\
                                 &= \left| \pi - 3.1416 \right| \\
                                 &= \left|-0.000007346410206832132\right| = 0.000007346410206832132 \\
                                 &\approx 0.00000735
            \end{aligned}
            \]
    
            \[
            \begin{aligned}
                error_{relativo} &= \left| \frac{p - p^{*}}{p}\right| \\
                                 &= \frac{0.00000735}{\pi} \\
                                 &= 0.0000023395776634508616 \\
                                 &\approx 0.00000234
            \end{aligned}
        \]
        \item \(p = e, p^{*} = 2.718\)
        \[
            \begin{aligned}
                error_{absoluto} &= \left| p - p^{*}\right| \\
                                 &= \left| e - 2.718 \right| \\
                                 &= \left|0.0002818284590451192\right| = 0.0002818284590451192 \\
                                 &\approx 0.0003
            \end{aligned}
            \]
    
            \[
            \begin{aligned}
                error_{relativo} &= \left| \frac{p - p^{*}}{p}\right| \\
                                 &= \frac{0.0003}{e} \\
                                 &= 0.00011036383235143269\\
                                 &\approx 0.0001104
            \end{aligned}
        \]
        \item \(p = \sqrt{2}, p^{*} = 1.414\)
        \[
            \begin{aligned}
                error_{absoluto} &= \left| p - p^{*}\right| \\
                                 &= \left| \sqrt{2} - 1.414 \right| \\
                                 &= \left|0.00021356237309522186\right| = 0.00021356237309522186 \\
                                 &\approx 0.0002136
            \end{aligned}
            \]
    
            \[
            \begin{aligned}
                error_{relativo} &= \left| \frac{p - p^{*}}{p}\right| \\
                                 &= \frac{0.0002136}{\sqrt{2}} \\
                                 &= 0.00015103800846144654\\
                                 &\approx 0.00015104
            \end{aligned}
        \]
    \end{enumerate}
    \item Calcule los errores absoluto y relativo en las aproximaciones de \(p\) por \(p^{*}\).
    \begin{enumerate}[label=\alph*.]
        \item \(p = e^{10}, p^{*} = 22000\)
        \[
            \begin{aligned}
                error_{absoluto} &= \left| p - p^{*}\right| \\
                                 &= \left| e^{10} - 22000 \right| \\
                                 &= \left|26.465794806703343\right| = 26.465794806703343 \\
                                 &\approx 26.4658
            \end{aligned}
            \]
    
            \[
            \begin{aligned}
                error_{relativo} &= \left| \frac{p - p^{*}}{p}\right| \\
                                 &= \frac{26.4658}{e^{10}} \\
                                 &= 0.0012015454611079724\\
                                 &\approx 0.0012015455
            \end{aligned}
        \]
        \item \(p = 10^{\pi}, p^{*} = 1400\)
        \[
            \begin{aligned}
                error_{absoluto} &= \left| p - p^{*}\right| \\
                                 &= \left| 10^{\pi} - 1400 \right| \\
                                 &= \left|-14.544268632989315\right| = 14.544268632989315 \\
                                 &\approx 14.54427
            \end{aligned}
            \]
    
            \[
            \begin{aligned}
                error_{relativo} &= \left| \frac{p - p^{*}}{p}\right| \\
                                 &= \frac{14.54427}{10^{\pi}} \\
                                 &= 0.010497823691305794\\
                                 &\approx 0.0105
            \end{aligned}
        \]
        \item \(p = 8!, p^{*} = 39900\)
        \[
            \begin{aligned}
                error_{absoluto} &= \left| p - p^{*}\right| \\
                                 &= \left| 8! - 39900 \right| \\
                                 &= \left|420\right| = 420
            \end{aligned}
            \]
    
            \[
            \begin{aligned}
                error_{relativo} &= \left| \frac{p - p^{*}}{p}\right| \\
                                 &= \frac{420}{8!} \\
                                 &= 0.01041\overline{6}\\
                                 &\approx 0.01042
            \end{aligned}
        \]
        \item \(p = 9!, p^{*} = \sqrt{18\pi}\left( \frac{9}{e}\right)^{9}\)
        \[
            \begin{aligned}
                error_{absoluto} &= \left| p - p^{*}\right| \\
                                 &= \left| 9! - \sqrt{18\pi}\left( \frac{9}{e}\right)^{9} \right| \\
                                 &= \left|3343.1271580516477\right| \approx 3343.13
            \end{aligned}
            \]
    
            \[
            \begin{aligned}
                error_{relativo} &= \left| \frac{p - p^{*}}{p}\right| \\
                                 &= \frac{3343.13}{9!} \\
                                 &= 0.009212770061728395\approx 0.01
            \end{aligned}
        \]
    \end{enumerate}
    \item Encuentro el intervalo más largo en el que se debe encontrar \(p^{*}\) para aproximarse a \(p\) con error relativo máximo de \(10^{-4}\) para cada valor de \(p\).

    \[
    \begin{aligned}
        error_{relativo} &\leq 10^{-4} \\
        \left|\frac{p-p^{*}}{p}\right| &\leq 10^{-4} \\
        \left|p-p^{*}\right| &\leq 10^{-4} \cdot |p| \\
        p^{*} \in [p-10^{-4}\cdot|p|& ,  p+10^{-4}\cdot|p|]
    \end{aligned}
    \]
    \begin{enumerate}[label=\alph*.]
        \item \(\pi\)
        \[
            \begin{aligned}
                p &= \pi \\   
                p^{*} \in [\pi-10^{-4}\cdot \pi& ,  \pi+10^{-4}\cdot\pi] \\
                p^{*} \in [3.141278494324434& ,  3.141906812855152]
            \end{aligned}
        \]

        \item \(e\)
        \[
            \begin{aligned}
                p &= e \\   
                p^{*} \in [e-10^{-4}\cdot e& ,  e+10^{-4}\cdot e] \\
                p^{*} \in [2.718010000276199& ,  2.718553656641891]
            \end{aligned}
        \]
        \item \(\sqrt{2}\)
        \[
            \begin{aligned}
                p &= \sqrt{2} \\   
                p^{*} \in [\sqrt{2}-10^{-4}\cdot \sqrt{2}& ,  \sqrt{2}+10^{-4}\cdot \sqrt{2}] \\
                p^{*} \in [1.4140721410168577& ,  1.4143549837293325]
            \end{aligned}
        \]
        \item \(\sqrt[3]{7}\)
        \[
            \begin{aligned}
                p &= \sqrt[3]{7} \\   
                p^{*} \in [\sqrt[3]{7}-10^{-4}\cdot \sqrt[3]{7}& ,  \sqrt[3]{7}+10^{-4}\cdot \sqrt[3]{7}] \\
                p^{*} \in [1.9127398896541117& ,  1.9131224758906662]
            \end{aligned}
        \]
    \end{enumerate}
    \item El número \(e\) se puede definir por medio de \(e = \sum_{n = 0}^{\infty} \left( \frac{1}{n!} \right)\), donde \(n! = n(n-1) \dots 2 \cdot 1\) para \(n \neq 0\) y \(0! = 1\). Calcule los errores absoluto y relativo en la siguiente aproximación de \(e\): 
    \begin{enumerate}[label=\alph*.]
        \item \(\sum_{n=0}^{5} \left( \frac{1}{n!}\right)\)
        \[
        \sum_{n = 0}^{5} \left( \frac{1}{n!}\right) = \frac{1}{0!} + \frac{1}{1!} + \frac{1}{2!} + \frac{1}{3!} + \frac{1}{4!} + \frac{1}{5!} = 2.71\overline{6} \approx 2.72
        \]
        \[
            \begin{aligned}
                error_{absoluto} &= \left| e - \sum_{n = 0}^{5} \left( \frac{1}{n!}\right)\right| \\
                                 &= \left| e - 2.72 \right| = |-0.0017181715409551046| \\
                                 &= 0.0017181715409551046 \approx 0.00172        
            \end{aligned}
        \]
        \[
            \begin{aligned}
                error_{relativo} &= \left| \frac{e - \sum_{n = 0}^{5} \left( \frac{1}{n!}\right)}{e}\right| \\
                                 &= \frac{0.00172}{e} = 0.0006327526388148809 \approx 0.00633        
            \end{aligned}
        \]  
        \item \(\sum_{n=0}^{10} \left( \frac{1}{n!}\right)\)
        \[
            \sum_{n = 0}^{10} \left( \frac{1}{n!}\right) = \frac{1}{0!} + \frac{1}{1!} + \frac{1}{2!} + \frac{1}{3!} + \frac{1}{4!} + \frac{1}{5!} + \frac{1}{6!} + \frac{1}{7!} + \frac{1}{8!} + \frac{1}{9!} + \frac{1}{10!} \approx 2.7182818
            \]
            \[
                \begin{aligned}
                    error_{absoluto} &= \left| e - \sum_{n = 0}^{10} \left( \frac{1}{n!}\right)\right| \\
                                     &= \left| e - 2.7182818 \right| = |0.000000028459044898454522| \\
                                     &= 0.000000028459044898454522 \approx 0.00000002846         
                \end{aligned}
            \]
            \[
                \begin{aligned}
                    error_{relativo} &= \left| \frac{e - \sum_{n = 0}^{10} \left( \frac{1}{n!}\right)}{e}\right| \\
                                     &= \frac{0.00000002846}{e} = 0.000000010469848895739249 \approx 0.00000001047        
                \end{aligned}
        \]
    \end{enumerate}

    \item Suponga que dos puntos (\(x_0\),\(y_0\)) y (\(x_1\),\(y_1\)) se encuentran en línea recta con \(y_1 \neq y_0\). Existen dos fórmulas para encontrar la intersección \(x\) de la línea:
    \[x = \frac{x_0y_1 - x_1y_0}{y_1-y_0} \quad \text{y} \quad x = x_0 - \frac{(x_1-x_0)y_0}{y_1-y_0}\]
    
    \begin{enumerate}[label=\alph*.]
        \item Use los datos \((x_0,y_0) = (1.31, 3.24)\) y \((x_1, y_1) = (1.93, 4.76)\) y la aritmética de redondeo de tres dígitos para calcular la intersección con \(x\) de ambas maneras. ¿Cuál método es mejor y por qué?
    \end{enumerate}
    
    \[
    \begin{aligned}
        x_a &= \frac{x_0y_1 - x_1y_0}{y_1-y_0} \\
          &= \frac{1.31 \cdot 4.76 - 1.93 \cdot 3.24}{4.76 - 3.24} \\
          &= -0.011578947368421534 \approx -0.0116
    \end{aligned}
    \]
    \[
    \begin{aligned}
        x_b &=  x_0 - \frac{(x_1-x_0)y_0}{y_1-y_0}\\
          &= 1.31 - \frac{(1.93 - 1.31) \cdot 3.24}{4.76 - 3.24} \\
          &= -0.011578947368421355 \approx -0.0116
    \end{aligned}
    \]

    Las dos formulas dan el mismo resultado. Entre las dos la segunda formula \(x_b\) da menos errores ya que no tiene tantas multiplicaciones, lo que la hace mas efectiva.
\end{enumerate}

\section*{CONCLUCIONES}
\begin{itemize}
    \item {Comprender los errores nos ayuda a identificar qué tanto nos estamos acercando al valor real de una operación.}
    \item {Se comprobó que las formulas que incluyen más multiplicaciones son mas propensas a errores.} 
\end{itemize}

\section*{RECOMENDACIONES}

Usar error absoluto y relativo para evaluar los resultados, ya que cada uno aporta
imformación única.
El absoluto da el valor de la diferencia entre el valor exacto y la aproximación.
El relativo da una comparación entre el valor del error absoluto y el valor exacto.
\renewcommand{\refname}{\MakeUppercase{REFERENCIAS}}
\bibliographystyle{IEEEtran}
\bibliography{inFiles/References/references.bib}


\end{document}
