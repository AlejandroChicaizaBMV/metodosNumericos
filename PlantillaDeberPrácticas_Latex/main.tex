\documentclass[12pt]{article}
\usepackage[spanish]{babel}
\usepackage{geometry}
\geometry{a4paper, margin=1in}
\usepackage{graphicx}
\usepackage{xcolor}
\usepackage{titlesec}
\usepackage{parskip}
\usepackage{multicol}
\usepackage{cite}
\usepackage{listings}
\usepackage{color}
\usepackage{amsmath}
\usepackage{enumitem}

\lstset{
  language=Python,
  basicstyle=\ttfamily\small,
  keywordstyle=\color{blue},
  commentstyle=\color{gray},
  stringstyle=\color{red},
  breaklines=true,
  showstringspaces=false
}


\definecolor{highlight}{RGB}{255, 255, 0}

\titleformat{\section}{\normalfont\Large\bfseries}{\thesection}{1em}{}
\titleformat{\subsection}{\normalfont\large\bfseries}{\thesubsection}{1em}{}

\begin{document}

% Logos
\begin{minipage}{0.45\textwidth}
    \includegraphics[width=0.4\textwidth]{inFiles/Figures/epnLogo.jpg}
\end{minipage}
\hfill
\begin{minipage}{0.45\textwidth}
    \raggedleft
    \includegraphics[width=0.4\textwidth]{inFiles/Figures/FIS_logo.jpg}
\end{minipage}

\vspace{0.5cm}

% Títulos principales
\begin{center}
    \textbf{ESCUELA POLITÉCNICA NACIONAL}\\[0.2cm]
    \textbf{FACULTAD DE INGENIERÍA DE SISTEMAS}\\[0.2cm]
    \textbf{INGENIERÍA EN CIENCIAS DE LA COMPUTACIÓN}
\end{center}

\vspace{0.5cm}
\hrule
\vspace{0.5cm}

% Datos principales
\noindent\textbf{PERÍODO ACADÉMICO:} 2025-A\\[0.2cm]
\noindent\textbf{ASIGNATURA:} ICCD412 Métodos Numéricos \hfill \textbf{GRUPO:} GR2\\[0.2cm]
\noindent\textbf{TIPO DE INSTRUMENTO:} Práctica2\\[0.2cm]
\noindent\textbf{FECHA DE ENTREGA LÍMITE:} {04/05/2025}\\[0.2cm]
\noindent\textbf{ALUMNO:} {Sebastián Chicaiza}

\vspace{0.5cm}
\hrule
\vspace{1cm}


% Secciones
\section*{TEMA}

\begin{center}
    \Large\textbf{chuuuupaloooo}
\end{center}
\vspace{0.5cm}

\section*{OBJETIVOS}
\begin{itemize}
    \item {Comprender el procedimiento para la conversión de números en decimal a formato IEEE 754 de 64 bits.}
    \item { Analizar el error relativo resultante de la conversión.}
\end{itemize}
\vspace{0.5cm}
\section*{MARCO TEÓRICO}
El estándar IEEE 754 define cómo representar números reales en computadoras usando formatos de punto flotante.
Los tipos más comunes son float y double. El valor cero no puede representarse en formato normalizado, por lo que se usa un patrón especial de bits en ceros. También existe el -0, pero tanto 0 como -0 se consideran iguales al compararlos\cite{microsoftIEEE754}.

El número representado en formato IEEE 754 de 64 bits esá dado por:
\[x = (-1)^{Signo} \cdot (1+Mantisa) \cdot 2^{exponente - 1023}\]\cite{bui1999design}
\vspace{0.5cm}

\section*{DESARROLLO}

Pasar el número \(263.3_{10}\) al formato IEEE 754 de 64 bits, una vez en binario pasarlo a decimal y calcular el error relativo a 3 cifras de redondeo.

\subsection*{Transformación a IEE 754}
Signo positivo (0)

Conversión a binario:
    
    \[
        \begin{aligned}
            \frac{263}{2}  = residuo(1) \\
            \frac{131}{2}  = residuo(1) \\
            \frac{65}{2}  = residuo(1) \\
            \frac{32}{2}  = residuo(0) \\
            \frac{16}{2}  = residuo(0) \\
            \frac{8}{2}  = residuo(0) \\
            \frac{4}{2}  = residuo(0) \\
            \frac{2}{2}  = residuo(0) \\
            \frac{1}{2}  = residuo(1) \\
            263_{10} = 100000111_2
        \end{aligned}
    \]

    \[
        \begin{aligned}
            0.3 \cdot 2 &= parteEntera(0) \\
            0.6 \cdot 2 &= parteEntera(1) \\
            0.2 \cdot 2 &= parteEntera(0) \\
            0.4 \cdot 2 &= parteEntera(0) \\
            0.8 \cdot 2 &= parteEntera(1) \\
            0.6 \cdot 2 &= parteEntera(1) \quad Se\quad repite\\
            0.3_{10} &= 0.0\overline{1001}_2
        \end{aligned}
    \]

\[ 263.3_{10} = 100000111.0\overline{1001}_2\]

Mantisa normalizada:
\[100000111.0\overline{1001}_2 = 1.000001110\overline{1001}_2 \cdot 2^{8}\]

Mantisa:
\[Mantisa = 000001110\overline{1001}_2\]
Exponente: 
\[Exponente_{Sesgado} = 1023 + exponenteReal\]
\[Exponente_{Sesgado} = 1023 + 8 = 1031\]

Conversión del Exponente:

\[
    \begin{aligned}
        \frac{1031}{2}  = residuo(1) \\
        \frac{515}{2}  = residuo(1) \\
        \frac{257}{2}  = residuo(1) \\
        \frac{128}{2}  = residuo(0) \\
        \frac{64}{2}  = residuo(0) \\
        \frac{32}{2}  = residuo(0) \\
        \frac{16}{2}  = residuo(0) \\
        \frac{8}{2}  = residuo(0) \\
        \frac{4}{2}  = residuo(0) \\
        \frac{2}{2}  = residuo(0) \\
        \frac{1}{2}  = residuo(1) \\
        1031_{10} = 10000000111_2
    \end{aligned}
\]

Exponente en 11 bits:
\[Exponente = 10000000111_2\]

\begin{center}
    \begin{tabular}{|c|c|c|}
    \hline
    \textbf{Signo} & \textbf{Exponente (11 bits)} & \textbf{Mantisa (52 bits)} \\ \hline
    0 & 10000000111 & 0000011101001100110011001100110011001100110011001100 \\ \hline
    \end{tabular}
\end{center}

\subsection*{Transformación a decimal}

\[x = (-1)^{Signo} \cdot (1+Mantisa) \cdot 2^{exponente - 1023}\]
\[x = (-1)^{0} \cdot (1+0.028515338897705078) \cdot 2^{1031 - 1023}\]
\[x = 263.2999267578125 \]

\subsection*{Cálculo de error relativo}
\[p = 263.3 \quad \text{y} \quad p^{*} = 263.2999267578125\]
\[
\begin{aligned}
    error_{relativo} &= \left| \frac{p - p^{*}}{p} \right|\\
                     &= \left| \frac{263.3 - 263.2999267578125}{263.3} \right|\\
                     &= 0.000000278170100688829 \approx 0.278 \cdot 10^{-6}
\end{aligned}
\]  
\section*{CONCLUCIONES}
\begin{itemize}
    \item {La conversion a formato IEEE 754 de 64 bits permite representar una gran cantidad de valores con alta presición.}
    \item {El resultado del cálculo del error relativo de la conversión demuestra la presición de este formato.} 
\end{itemize}

\section*{RECOMENDACIONES}

Tener en cuenta el valor del error relativo al utilizar el formato IEEE 754 para representar números.
Especialmente cuando se trata de resultados en el campo científico, donde se requiere de gran presición.
\renewcommand{\refname}{\MakeUppercase{REFERENCIAS}}
\bibliographystyle{IEEEtran}
\bibliography{inFiles/References/references.bib}


\end{document}
