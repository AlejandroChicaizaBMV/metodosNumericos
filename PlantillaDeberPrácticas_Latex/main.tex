\documentclass[12pt]{article}
\usepackage[spanish]{babel}
\usepackage{geometry}
\geometry{a4paper, margin=1in}
\usepackage{graphicx}
\usepackage{xcolor}
\usepackage{titlesec}
\usepackage{parskip}
\usepackage{multicol}
\usepackage{cite}
\usepackage{listings}
\usepackage{color}
\usepackage{amsmath}
\usepackage{enumitem}

\lstset{
  language=Python,
  basicstyle=\ttfamily\small,
  keywordstyle=\color{blue},
  commentstyle=\color{gray},
  stringstyle=\color{red},
  breaklines=true,
  showstringspaces=false
}


\definecolor{highlight}{RGB}{255, 255, 0}

\titleformat{\section}{\normalfont\Large\bfseries}{\thesection}{1em}{}
\titleformat{\subsection}{\normalfont\large\bfseries}{\thesubsection}{1em}{}

\begin{document}

% Logos
\begin{minipage}{0.45\textwidth}
    \includegraphics[width=0.4\textwidth]{inFiles/Figures/epnLogo.jpg}
\end{minipage}
\hfill
\begin{minipage}{0.45\textwidth}
    \raggedleft
    \includegraphics[width=0.4\textwidth]{inFiles/Figures/FIS_logo.jpg}
\end{minipage}

\vspace{0.5cm}

% Títulos principales
\begin{center}
    \textbf{ESCUELA POLITÉCNICA NACIONAL}\\[0.2cm]
    \textbf{FACULTAD DE INGENIERÍA DE SISTEMAS}\\[0.2cm]
    \textbf{INGENIERÍA EN CIENCIAS DE LA COMPUTACIÓN}
\end{center}

\vspace{0.5cm}
\hrule
\vspace{0.5cm}

% Datos principales
\noindent\textbf{PERÍODO ACADÉMICO:} 2025-A\\[0.2cm]
\noindent\textbf{ASIGNATURA:} ICCD412 Métodos Numéricos \hfill \textbf{GRUPO:} GR2\\[0.2cm]
\noindent\textbf{TIPO DE INSTRUMENTO:} Tarea6\\[0.2cm]
\noindent\textbf{FECHA DE ENTREGA LÍMITE:} {09/05/2025}\\[0.2cm]
\noindent\textbf{ALUMNO:} {Sebastián Chicaiza}

\vspace{0.5cm}
\hrule
\vspace{1cm}


% Secciones
\section*{TEMA}

\begin{center}
    \Large\textbf{Método de la secante}
\end{center}
\vspace{0.5cm}

\section*{OBJETIVOS}
\begin{itemize}
    \item {Aplicar los métodos de la secante y Newton-Rahson para encontrar valores aproximados a la raíz de una función no lineal.}
    \item {Comparar la presición y convergencia de los dos métodos aplicados.}
\end{itemize}
\vspace{0.5cm}
\section*{MARCO TEÓRICO}
En el método de la secante en lugar de obtener una sucesión de intervalos se
calcula una sucesión de números que aproximan el cero de la función

Se comienza con dos aproximaciones \(p_0\) y \(p_1\) a la raíz que se está buscando.
Primero se construyen las rectas secantes que pasan por los puntos \(p_0, f(p_0)\), \(p_1, f(p_1)\) y se calcula su corte con el eje X
y este corte lo llamaremos \(p_2\), después se evalúa \(f(p_2)\). El proceso se repetirá con \(p_1\) y \(p_2\) para calcular \(p_3\) y así sucesivamente \(p_4, p_5, \dots\, x_n\)\cite{vadilloecuaciones}.

\(p_n\) puede ser exprezado de la siguiente forma:
\[p_n = p_{n-1} - f(p_{n-1}) \cdot \frac{p_{n-1} - p_{n-2}}{f(p_{n-1})- f(p_{n-2})}\]

\section*{DESARROLLO}
Use el método de la secante para encontrar una solución para \(x = \cos{(x)}(f(x) = \cos{(x)} -x = 0)\) con
tolerancia tal que:
\[|p_n - p_{n-1}| < (tolerancia = 10^{-16})\]
y compare las aproximaciones con las determinadas en el ejemplo visto en clase, el cual aplica el método
de Newton, resuelva hasta llegar a la misma tolerancia para este método también.

Suponga que usamos \(p_0 = 0.5 \) y \(p_1 = \frac{\pi}{4} \), trabaje con 13 cifras decimales de redondeo. 

\begin{center}
\begin{tabular}{|c|c|c|c|}
\hline
\(p_0\) & \(p_1\) & \(p_n\) & \(f(p_0)\) \\ 
\hline
0.5 & 0.7853981633974 & 0.7363841388366 & 0.3775825618904 \\
0.7853981633974 & 0.7363841388366 & 0.7390581392139 & -0.0782913822108 \\
0.7363841388366 & 0.7390581392139 & 0.7390851493372 & 0.0045177185221  \\
0.7390581392139 & 0.7390851493372 & 0.7390851332151 & 0.0000451772159e  \\
0.7390851493372 & 0.7390851332151 & 0.7390851332152 & -0.000000026982  \\
0.7390851332151 & 0.7390851332152 & 0.7390851332152 & 0.0000000000001 \\
\hline
\end{tabular}
\end{center}


\begin{center}
    \begin{tabular}{|c|c|c|}
    \hline
    \(f(p_1)\) & \(f(p_n)\) & \textbf{TOL} \\ 
    \hline
    -0.0782913822108 & 0.0045177185221 & 0.0490140245608 \\
    0.0045177185221 & 0.0000451772159 & 0.0026740003773 \\
    4.51772159e-05 & -0.000000026982 & 0.0000270101233 \\
    -2.6982e-08 & 0.0000000000001 & 0.000000000000161221  \\
    0.0000000000001 & -0.0000000000001 & 0.00000000000001 \\
    -0.0000000000001 & -0.0000000000001 & 0.0 \\
    \hline
    \end{tabular}
\end{center}
\(Raiz aproximada \approx 0.7390851332152 \)
    
Resultados obtenidos con el método de Newton-Raphson:

\begin{center}
    \begin{tabular}{|c|c|c|c|c|}
    \hline
    \(p_{n-1}\) & \(p_n\) & \(f(p_{n-1})\) & \(f^{'}(p_{n-1})\) & \textbf{TOL}  \\
    \hline
    0.5 & 0.7552224171057 & 0.3775825618904 & -1.4794255386042 &  0.2552224171057 \\
    0.7552224171057 & 0.7391416661499 & -0.0271033118576 & -1.6854506317545 &  0.0160807509558 \\
    0.7391416661499 & 0.7390851339208 & -0.0000946153806 & -1.6736538107584 &  5.65322291e-5 \\
    0.7390851339208 & 0.7390851332151 & -1.1810E-9 & -1.6736120297047 &  7.057e-10 \\
    0.7390851332151 & 0.7390851332152 & 1E-13 & -1.6736120291832 &  1.e-13 \\
    0.7390851332152 & 0.7390851332151 & -0E-13 & -1.6736120291832 &  1.e-13 \\
    0.7390851332151 & 0.7390851332152 & 1E-13 & -1.6736120291832 &  1.e-13 \\
    0.7390851332152 & 0.7390851332151 & -0E-13 & -1.6736120291832 &  1.e-13 \\
    0.7390851332151 & 0.7390851332152 & 1E-13 & -1.6736120291832 &  1.e-13 \\
    0.7390851332152 & 0.7390851332151 & -0E-13 & -1.6736120291832 &  1.e-13 \\
    \hline        
    \end{tabular}
\end{center}

\(Raiz aproximada \approx 0.7390851332151 \)
\section*{CONCLUSIONES}
\begin{itemize}
    \item {los resultados 0.7390851332152 secante y 0.7390851332151 Newton-Raphson demuestra la alta precisión de ambas técnicas para esta función en particular.}
    \item {El método de Newton-Raphson suele tener una tasa de convergencia más rápida, sin embargo se ve limitada al tener que usar derivadas, lo que implica una serie de pasos extras.} 
\end{itemize}

\section*{RECOMENDACIONES}

\begin{itemize}
    \item Usar librerias como sympy para representar derivadas.
\end{itemize}

\renewcommand{\refname}{\MakeUppercase{REFERENCIAS}}
\bibliographystyle{IEEEtran}
\bibliography{inFiles/References/references.bib}


\end{document}
